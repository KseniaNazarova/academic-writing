%%%%%%%%%%%%%%%%%%%%%%%%%%%%%%%%%%%%%%%%%%%%%%%%%%%%%%%%%%%%%%%%%%%%%%%%%%%%%%%%
%2345678901234567890123456789012345678901234567890123456789012345678901234567890
%        1         2         3         4         5         6         7         8

\documentclass[a4paper, 10pt, conference]{ieeeconf} 

\title{\LARGE \bf
Visualization module solving logical problems for the FORMULA system
}


\author{Ksenia Panova% <-this % stops a space
}


\begin{document}

\maketitle
\thispagestyle{empty}
\pagestyle{empty}


%%%%%%%%%%%%%%%%%%%%%%%%%%%%%%%%%%%%%%%%%%%%%%%%%%%%%%%%%%%%%%%%%%%%%%%%%%%%%%%%
\begin{abstract}

lalala

\end{abstract}


%%%%%%%%%%%%%%%%%%%%%%%%%%%%%%%%%%%%%%%%%%%%%%%%%%%%%%%%%%%%%%%%%%%%%%%%%%%%%%%%
\section{INTRODUCTION}

We have designed the module of visualization of solutions obtained via the smt solver Formula. Solvers for satisfiability modulo theories (SMT) check the satisfiability of first-order formulas containing operations from various theories such as the booleans, bit-vectors, arithmetic. A thorough consideration of various methods of visualization was made. As a result of studies the declarative programming language for solutions of logic problems such as Einstein task or the problem of queens was developed.

\section{PROCEDURE FOR PAPER SUBMISSION}

\subsection{Selecting a Template (Heading 2)}
The developed system is designed for data visualization. The purpose of the study is to consider an information visualization solutions of logical problems derived from the FORMULA system, which is fundamentally different from existing visualization tools. The most obvious advantage of the new system is that the developed module operates with a much smaller amount of information, which saves time and expense of resources for the visualization process.
\\The main task of the investigation is to design an integrated module of visualization solutions and the results of logical tasks.
\\The developed module should maintain the following functions:
\begin{enumerate}
\item Integration module logical solution FORMULA tasks, analysis of the results of using it
\item Display of static solutions with support for the following types of visualization:
\begin{enumerate}
\item Table
\item Counts
\item Vectors
\end{enumerate}
\item Display of the process of solving problems by means of a dynamic visualization:
\begin{enumerate}
\item Moving objects
\item Time Management
\end{enumerate}
\end{enumerate}

\subsection{Figures and Tables}

\begin{table}[h]
\caption{An Example of a Table}
\label{table_example}
\begin{center}
\begin{tabular}{|c||c|}
\hline
One & Two\\
\hline
Three & Four\\
\hline
\end{tabular}
\end{center}
\end{table}


   \begin{figure}[thpb]
      \centering
      \framebox{\parbox{3in}{We suggest that you use a text box to insert a graphic (which is ideally a 300 dpi TIFF or EPS file, with all fonts embedded) because, in an document, this method is somewhat more stable than directly inserting a picture.
}}
      %\includegraphics[scale=1.0]{figurefile}
      \caption{Inductance of oscillation winding on amorphous
       magnetic core versus DC bias magnetic field}
      \label{figurelabel}
   \end{figure}
   

\section{CONCLUSIONS}

As a result of studies the integrated module of visualization solutions and the results of logical tasks was designed. Also visualization language was developed. For that rules and  a new programming language syntax  have been formulated, on the basis of which carried out the lexical, syntactic and semantic analysis. Graphical system module is also designed to visualize input data in a predetermined format, namely in the form of static and dynamic table moving objects.   

\addtolength{\textheight}{-12cm}   % This command serves to balance the column lengths
                                  % on the last page of the document manually. It shortens
                                  % the textheight of the last page by a suitable amount.
                                  % This command does not take effect until the next page
                                  % so it should come on the page before the last. Make
                                  % sure that you do not shorten the textheight too much.

%%%%%%%%%%%%%%%%%%%%%%%%%%%%%%%%%%%%%%%%%%%%%%%%%%%%%%%%%%%%%%%%%%%%%%%%%%%%%%%%



\begin{thebibliography}{99}

\bibitem{c1} M. Kaufmann, Information Visualization: Perception for Design, 2004.
\bibitem{c2} A. Aho, M. Lam, R. Sethi, LCompilers: Principles, Techniques, and Tools, 2008.
\bibitem{c3} M. Khan, S.S. Khan, Data and Information Visualization Methods, and
Interactive Mechanisms: A Survey, 2010.
\bibitem{c4} D. Kulikov, Means, solutions and approaches to data visualization in medical information systems, 2009

\end{thebibliography}

\end{document}
